\begin{juliacode}
using PyPlot
x = linspace(0, 2π, 200)
plot(x, sin(x))
\end{juliacode}
\begin{juliaout}
1-element Array{Any,1}:
 PyObject <matplotlib.lines.Line2D object at 0x7fbd66371c50>
\end{juliaout}
\begin{figure}[ht]
\center
\includegraphics[width=\linewidth]{figures/pyplot_formats_sin_fun_1.pdf}
\caption{sin(x) function.}
\label{fig:sin_fun}
\end{figure}

\begin{juliaout}
1-element Array{Any,1}:
 PyObject <matplotlib.lines.Line2D object at 0x7fbd66325a20>
\end{juliaout}
\begin{figure}[htpb]
\center
\includegraphics[width=\linewidth]{figures/pyplot_formats_2_1.pdf}
\caption{cos(x) function.}
\end{figure}

\begin{juliaout}
1-element Array{Any,1}:
 PyObject <matplotlib.lines.Line2D object at 0x7fbd6629cf28>
\end{juliaout}
\includegraphics[width=\linewidth]{figures/pyplot_formats_cos2_fun_1.pdf}

\begin{juliaterm}
julia> x = linspace(0, 2π, 200)
200-element LinSpace{Float64}:
 0.0,0.0315738,0.0631476,0.0947214,0.126295,…,6.18846,6.22004,6.25161,6.28319

julia> plot(x, sin(x))
1-element Array{Any,1}:
 PyObject <matplotlib.lines.Line2D object at 0x7fbd600ca588>

julia> y = 20
20

julia> plot(x, cos(x))
1-element Array{Any,1}:
 PyObject <matplotlib.lines.Line2D object at 0x7fbd600f4c18>

\end{juliaterm}
\includegraphics[width=\linewidth]{figures/pyplot_formats_4_1.pdf}

\begin{juliacode}
x = randn(100, 100)
contourf(x)
\end{juliacode}
\begin{juliaout}
PyObject <matplotlib.contour.QuadContourSet object at 0x7fbd600ba9b0>
\end{juliaout}
\includegraphics[width=15cm]{figures/pyplot_formats_5_1.pdf}
