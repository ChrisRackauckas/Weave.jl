




\begin{juliacode}
using PyPlot
x = linspace(0, 2π, 200)
plot(x, sin(x))
\end{juliacode}
\begin{figure}[ht]
\center
\includegraphics[width=\linewidth]{figures/pyplot_formats_sin_fun_1.pdf}
\caption{sin(x) function.}
\label{fig:sin_fun}
\end{figure}



\begin{figure}[htpb]
\center
\includegraphics[width=\linewidth]{figures/pyplot_formats_2_1.pdf}
\caption{cos(x) function.}
\end{figure}



\includegraphics[width=\linewidth]{figures/pyplot_formats_cos2_fun_1.pdf}



\begin{juliaterm}
julia> x = linspace(0, 2π, 200)
200-element Array{Float64,1}:
 0.0      
 0.0315738
 0.0631476
 0.0947214
 0.126295 
 0.157869 
 0.189443 
 0.221017 
 0.25259  
 0.284164 
 ⋮        
 6.03059  
 6.06217  
 6.09374  
 6.12532  
 6.15689  
 6.18846  
 6.22004  
 6.25161  
 6.28319  

julia> plot(x, sin(x))
1-element Array{Any,1}:
 PyObject <matplotlib.lines.Line2D object at 0x7f9541e564d0>

julia> y = 20
20

julia> plot(x, cos(x))
1-element Array{Any,1}:
 PyObject <matplotlib.lines.Line2D object at 0x7f9541e56750>

\end{juliaterm}
\includegraphics[width=\linewidth]{figures/pyplot_formats_4_1.pdf}



\begin{juliacode}
x = randn(100, 100)
contourf(x)
\end{juliacode}
\includegraphics[width=15cm]{figures/pyplot_formats_5_1.pdf}


